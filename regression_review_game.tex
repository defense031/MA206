\documentclass[12pt]{article}
\usepackage{geometry}
\geometry{a4paper, margin=1in}
\usepackage{graphicx} % For including images
\usepackage{fancyhdr} % For headers and footers
\usepackage{hyperref}
\usepackage{caption}
\usepackage{booktabs}
\usepackage{amsmath}
\usepackage{xcolor}
\usepackage{pgfplots}
\usepackage{subcaption}
\usepackage{pdflscape}
\usepackage{float}
\usepackage{placeins}
\renewcommand\arraystretch{1.3}
\pgfplotsset{compat=1.17}
\usepackage{tikz}

\begin{document}
% Header block
\noindent
\begin{minipage}[t]{0.075\textwidth}
    \vspace{0pt}
    \includegraphics[width=1.6cm]{figures/Math_Mule.png}
\end{minipage}
\hfill
\begin{minipage}[t]{0.9\textwidth}
    \vspace{0pt}
    \begin{tabular}{p{0.55\textwidth} p{0.4\textwidth}}
        \textbf{\Large MA206} & \hfill AY25-2 \\
        Probability \& Statistics & \hfill Lesson 21 \\
        Lesson: Linear Regression Review & \hfill Cadet: \underline{\hspace{2cm}}
    \end{tabular}
\end{minipage}

\vspace{1em}
\hrule
\vspace{0.5cm}

\section*{Lesson Objectives}
\begin{enumerate}
    \item Understand the impact of interaction effects on a linear model
    \item Review linear regression concepts through hands-on problem solving
    \item Practice hypothesis testing and model interpretation
\end{enumerate}

\hrule
\vspace{0.5cm}

% ==================== PART 1: INTERACTION EFFECTS LESSON (10-15 min) ====================
\section*{Part 1: Introduction to Interaction Effects}

\subsection*{Interaction Effects: Conceptual Overview}

\noindent
\textbf{Motivating Example:} A tutoring program may boost low‑performers' test scores far more than high‑performers'. This unequal benefit is an \textit{interaction} effect.

\vspace{0.25cm}

\noindent
\textbf{What Is an Interaction?}
\begin{itemize}
  \item Occurs when one predictor's effect on \(Y\) changes based on another predictor.
  \item Captures synergy between variables.
        \begin{itemize}
            \item Amplifying
            \item Dampening
        \end{itemize}
\end{itemize}

\vspace{0.25cm}

\noindent
\textbf{Visual Comparison:}
\begin{itemize}
  \item \textbf{Before Interaction:} Parallel slopes → same effect of \(X\) across groups.
  \item \textbf{After Interaction:} Divergent slopes → effect of \(X\) depends on group.
\end{itemize}

\begin{figure}[H]
  \centering
  \includegraphics[width=0.9\textwidth]{figures/interaction_before_after.png}
  \caption{Left: No interaction.     \hspace{1.5cm}        Right: With interaction.}
\end{figure}

\vspace{0.25cm}

\noindent
\textbf{General Form:}
\[
\widehat y = \hat{\beta}_0 + \hat{\beta}_1 x_1 + \hat{\beta}_2 x_2 + \hat{\beta}_3 (x_1 \times x_2)
\]
\begin{itemize}
  \item \(\hat{\beta}_3\) $\neq$ 0 indicates an interaction may be present.
\end{itemize}

\vspace{0.25cm}

\noindent
\textbf{Interpretation of \(\hat{\beta}_3\):}
\begin{itemize}
  \item Positive: predictors amplify each other.
  \item Negative: one predictor weakens the other.
\end{itemize}

\vspace{0.5cm}

\noindent
\textbf{Why It Matters:}
\begin{itemize}
  \item Reveals conditional relationships that additive models miss.
  \item Helps identify subgroups for whom predictors have stronger or weaker effects.
  \item Improves predictive accuracy by capturing the interplay between variables.
\end{itemize}

\subsection*{Example: Attendance × CorpsSquad}
An \textbf{interaction} means the effect of one predictor depends on another. Here: "Does the benefit of higher attendance on TEE differ for Corps Squad cadets versus non‑Corps Squad cadets?"

\begin{figure}[H]
  \centering
  \begin{subfigure}[b]{0.45\textwidth}
    \includegraphics[width=\textwidth]{figures/attendance_interaction_plot.png}
    \caption{Predicted TEE vs Attendance}
  \end{subfigure}
  \hfill
  \begin{subfigure}[b]{0.45\textwidth}
    \includegraphics[width=\textwidth]{figures/attendance_slope_plot.png}
    \caption{Attendance Effect (Slope) by Group}
  \end{subfigure}
  \caption{Interaction shows Corps Squad cadets gain more TEE points per \% attendance than non‑Corps cadets.}
\end{figure}
\FloatBarrier

\noindent
\textbf{Takeaway:} The red line's steeper slope indicates Corps Squad cadets benefit more from improved attendance — a possible interaction effect!

\newpage

% ==================== PART 2: SCAVENGER HUNT GAME (45-50 min) ====================
\section*{Part 2: Linear Regression Review Scavenger Hunt}

\subsection*{Game Rules and Scoring System}

\textbf{Overview:} You will work through \emph{five} regression problems. Most problems have three stages, but the final problem has six stages:
\begin{itemize}
    \item \textbf{Stage 1:} Identify the regression concept being tested
    \item \textbf{Stage 2:} Examine regression output or data
    \item \textbf{Stage 3:} Perform calculations or state hypotheses
\end{itemize}

\textbf{Scoring:}
\begin{itemize}
    \item Each stage of each problem is worth \textbf{100 points} if you answer correctly on the first attempt.
    \item If your first attempt is \emph{incorrect}, you can only earn \textbf{50 points} for that stage.
    \item Hints cost 25 points each.
\end{itemize}

\textbf{Flow of the Game:}
\begin{enumerate}
    \item Complete all three stages of each problem.
    \item Check your answer with CPT Semmel after each stage to receive your next clue.
    \item Add up your points at the end to see your total out of \textbf{1800 points}!
\end{enumerate}

\vspace{1cm}

\subsection*{Reference: LINE Conditions for Linear Regression}
Remember that valid linear regression requires checking:
\begin{itemize}
    \item \textbf{L}inearity: The relationship between predictors and response is linear
    \item \textbf{I}ndependence: Observations are independent
    \item \textbf{N}ormality: Residuals are approximately normally distributed
    \item \textbf{E}qual variance: Residuals have constant variance (homoskedasticity)
\end{itemize}

\newpage

%%%%%%%%%%%%%%%%%%%%%%%%%%%%%% PROBLEM 1 %%%%%%%%%%%%%%%%%%%%%%%%%%%%%%
\begin{center}
  {\Large \textbf{Problem 1: Simple Linear Regression Basics}}
\end{center}

\vspace{1cm}
\noindent
\textbf{Stage 1: Context \& Concept}

A sports scientist studies the relationship between hours of weekly training and race completion time (in minutes) for marathon runners. After collecting data from 25 runners, she fits a simple linear regression model:

\[
\widehat{\text{Time}} = 245 - 3.2 \times \text{Hours}
\]

\noindent
\textbf{Question:} What does the slope coefficient $-3.2$ mean in practical terms?

\vspace{3cm}

\noindent
\textbf{Stage 2: Data Interpretation}

Using the model above, predict the marathon completion time for a runner who trains 15 hours per week.

\vspace{3cm}

\noindent
\textbf{Stage 3: Residual Calculation}

Suppose a runner who trains 15 hours per week actually completed the marathon in 190 minutes. Calculate the residual for this observation.

\vspace{0.5cm}
\noindent
Residual = Observed $-$ Predicted = \underline{\hspace{3cm}}

\newpage

%%%%%%%%%%%%%%%%%%%%%%%%%%%%%% PROBLEM 2 %%%%%%%%%%%%%%%%%%%%%%%%%%%%%%
\begin{center}
  {\Large \textbf{Problem 2: Hypothesis Testing for Slope}}
\end{center}

\vspace{1cm}
\noindent
\textbf{Stage 1: Setting Up Hypotheses}

An economist wants to test whether family income (\$1000's) has a significant effect on annual spending on education (\$). She collects data from 40 families and fits a regression model. She wants to test if the slope is significantly different from zero.

\vspace{0.5cm}
\noindent
\textbf{Question:} Write the null and alternative hypotheses for testing whether income has an effect on education spending.

\begin{align*}
H_0: &\underline{\hspace{5cm}} \\
H_a: &\underline{\hspace{5cm}}
\end{align*}

\vspace{2cm}

\noindent
\textbf{Stage 2: Understanding Output}

The regression output shows:
\begin{itemize}
    \item $\hat{\beta}_1 = 85.5$ (slope estimate)
    \item $SE(\hat{\beta}_1) = 22.3$ (standard error of slope)
    \item Sample size: $n = 40$
\end{itemize}

\vspace{0.5cm}
\noindent
\textbf{Question:} Calculate the t-statistic for testing $H_0: \beta_1 = 0$.

\[
t = \frac{\hat{\beta}_1 - 0}{SE(\hat{\beta}_1)} = \underline{\hspace{3cm}}
\]

\vspace{2cm}

\noindent
\textbf{Stage 3: Making a Decision}
\noindent
Using $df = n - 2 = 38$ and a significance level of $\alpha = 0.05$, the critical value is approximately $t^* = 2.024$.

\vspace{0.5cm}
\noindent
\textbf{Question:} Based on your t-statistic, do you reject or fail to reject $H_0$? What does this mean in context?

\vspace{3cm}

\newpage

%%%%%%%%%%%%%%%%%%%%%%%%%%%%%% PROBLEM 3 %%%%%%%%%%%%%%%%%%%%%%%%%%%%%%
\begin{center}
  {\Large \textbf{Problem 3: Confidence Intervals and $R^2$}}
\end{center}

\vspace{1cm}
\noindent
\textbf{Stage 1: Confidence Interval for Slope}

A biologist studies how temperature (°C) affects bacterial growth rate (colonies/hour). Her regression output shows:
\begin{itemize}
    \item $\hat{\beta}_1 = 1.85$ colonies per hour per degree
    \item $SE(\hat{\beta}_1) = 0.42$
    \item $n = 30$ observations, so $df = 28$
    \item $t^*_{0.025, 28} = 2.048$ (for 95\% CI)
\end{itemize}

\vspace{0.5cm}
\noindent
\textbf{Question:} Calculate a 95\% confidence interval for the true slope $\beta_1$.

\[
\hat{\beta}_1 \pm t^* \times SE(\hat{\beta}_1) = \underline{\hspace{5cm}}
\]

\vspace{2cm}

\noindent
\textbf{Stage 2: Interpreting $R^2$}

The regression output also shows $R^2 = 0.73$.

\vspace{0.5cm}
\noindent
\textbf{Question:} What does this $R^2$ value tell you about the model? Write a complete sentence.

\vspace{3cm}

\noindent
\textbf{Stage 3: Understanding Adjusted $R^2$}

If the biologist adds more predictors to the model, which will always increase: $R^2$, Adjusted $R^2$, or both? Why does Adjusted $R^2$ exist?

\vspace{3cm}

\newpage

%%%%%%%%%%%%%%%%%%%%%%%%%%%%%% PROBLEM 4 %%%%%%%%%%%%%%%%%%%%%%%%%%%%%%
\begin{center}
  {\Large \textbf{Problem 4: Multiple Regression Interpretation}}
\end{center}

\vspace{1cm}
\noindent
\textbf{Stage 1: Model Specification}

A real estate analyst wants to predict house price (in \$1000's) using:
\begin{itemize}
    \item Square footage (SqFt)
    \item Number of bedrooms (Beds)
    \item Distance to downtown (Miles)
\end{itemize}

\vspace{0.5cm}
\noindent
\textbf{Question:} Write the full population regression model with proper notation.

\[
\text{Price} = \underline{\hspace{8cm}}
\]

\vspace{2cm}

\noindent
\textbf{Stage 2: Coefficient Interpretation}

After fitting the model, the estimated equation is:
\[
\widehat{\text{Price}} = 50.3 + 0.12 \times \text{SqFt} + 25.5 \times \text{Beds} - 8.3 \times \text{Miles}
\]

\vspace{0.5cm}
\noindent
\textbf{Question:} Interpret the coefficient $\hat{\beta}_{\text{Beds}} = 25.5$ in context. Make sure to include the phrase "holding all else constant."

\vspace{3cm}

\noindent
\textbf{Stage 3: Making a Prediction}

Use the fitted model to predict the price of a house with:
\begin{itemize}
    \item 2000 square feet
    \item 3 bedrooms
    \item 5 miles from downtown
\end{itemize}

\[
\widehat{\text{Price}} = \underline{\hspace{6cm}}
\]

\vspace{2cm}

\newpage

%%%%%%%%%%%%%%%%%%%%%%%%%%%%%% PROBLEM 5 %%%%%%%%%%%%%%%%%%%%%%%%%%%%%%
\begin{center}
  {\Large \textbf{Problem 5: Categorical Variables, Dummy Coding \& Interactions}}
\end{center}

\vspace{1cm}
\noindent
\textbf{Stage 1: Understanding Dummy Variables}

A study examines the relationship between years of experience and salary, but also accounts for whether employees work remotely or in-office. The researchers create a dummy variable:
\[
\text{Remote} = \begin{cases}
1 & \text{if employee works remotely} \\
0 & \text{if employee works in office}
\end{cases}
\]

They fit the model:
\[
\widehat{\text{Salary}} = 45000 + 2500 \times \text{Experience} + 7000 \times \text{Remote}
\]

\vspace{0.5cm}
\noindent
\textbf{Question:} What is the baseline category (reference group) in this model?

\vspace{2cm}

\noindent
\textbf{Stage 2: Interpretation}

Using the model above:

\vspace{0.5cm}
\noindent
\textbf{Question A:} What is the predicted salary for an in-office employee with 5 years of experience?

\vspace{2cm}

\noindent
\textbf{Question B:} What is the predicted salary for a remote employee with 5 years of experience?

\vspace{2cm}

\noindent
\textbf{Stage 3: Coefficient Meaning}

\textbf{Question:} In plain English, what does the coefficient $\hat{\beta}_{\text{Remote}} = 7000$ tell us?

\vspace{3cm}

\noindent
\textbf{Stage 4: Recognizing Interactions}

Recall the interaction lesson from Part 1. A researcher studying student test scores believes that study hours might have a different effect for students who attend tutoring versus those who don't.

\vspace{0.5cm}
\noindent
She fits two models:

\textbf{Model 1 (No Interaction):}
\[
\widehat{\text{Score}} = 60 + 2.5 \times \text{Hours} + 8 \times \text{Tutoring}
\]

\textbf{Model 2 (With Interaction):}
\[
\widehat{\text{Score}} = 55 + 1.8 \times \text{Hours} + 12 \times \text{Tutoring} + 1.2 \times (\text{Hours} \times \text{Tutoring})
\]

where Tutoring = 1 if student attended tutoring, 0 otherwise.

\vspace{0.5cm}
\noindent
\textbf{Question:} In Model 1, what does the slope of Hours represent for students who did NOT attend tutoring? What about for students who DID attend tutoring?

\vspace{3cm}

\noindent
\textbf{Stage 5: Understanding the Interaction Term}

Using Model 2:

\vspace{0.5cm}
\noindent
\textbf{Question A:} Write the fitted equation for students who DID NOT attend tutoring (Tutoring = 0).

\vspace{2cm}

\noindent
\textbf{Question B:} Write the fitted equation for students who DID attend tutoring (Tutoring = 1).

\vspace{2cm}

\noindent
\textbf{Stage 6: Interpretation and Hypothesis Testing}

\textbf{Question A:} Based on your equations from Stage 5, what is the slope of Hours for each group? Does the effect of study hours differ between groups?

\vspace{3cm}

\noindent
\textbf{Question B:} If we wanted to test whether the interaction term is statistically significant, what would be our null and alternative hypotheses?

\begin{align*}
H_0: &\underline{\hspace{5cm}} \\
H_a: &\underline{\hspace{5cm}}
\end{align*}

\vspace{2cm}
\noindent
\textbf{Congratulations!} You've completed the Linear Regression Review Scavenger Hunt!

\vspace{1cm}
\noindent
\textbf{Total Score: \underline{\hspace{2cm}} / 1800}

\newpage

\section*{Answer Key for CPT Semmel}

\subsection*{Problem 1: Simple Linear Regression Basics}
\begin{itemize}
    \item \textbf{Stage 1:} For each additional hour of weekly training, marathon completion time decreases by 3.2 minutes (on average).
    \item \textbf{Stage 2:} $\widehat{\text{Time}} = 245 - 3.2(15) = 245 - 48 = 197$ minutes
    \item \textbf{Stage 3:} Residual = $190 - 197 = -7$ minutes (model overestimated)
\end{itemize}

\subsection*{Problem 2: Hypothesis Testing for Slope}
\begin{itemize}
    \item \textbf{Stage 1:} $H_0: \beta_1 = 0$; $H_a: \beta_1 \neq 0$
    \item \textbf{Stage 2:} $t = 85.5/22.3 = 3.834$
    \item \textbf{Stage 3:} Since $|t| = 3.834 > 2.024$, reject $H_0$. There is significant evidence that income has an effect on education spending.
\end{itemize}

\subsection*{Problem 3: Confidence Intervals and $R^2$}
\begin{itemize}
    \item \textbf{Stage 1:} $1.85 \pm 2.048(0.42) = 1.85 \pm 0.860 = (0.99, 2.71)$ colonies per hour per degree
    \item \textbf{Stage 2:} 73\% of the variation in bacterial growth rate is explained by temperature.
    \item \textbf{Stage 3:} $R^2$ always increases (or stays same); Adjusted $R^2$ penalizes for adding predictors that don't improve the model enough.
\end{itemize}

\subsection*{Problem 4: Multiple Regression Interpretation}
\begin{itemize}
    \item \textbf{Stage 1:} $\text{Price} = \beta_0 + \beta_1 \text{SqFt} + \beta_2 \text{Beds} + \beta_3 \text{Miles} + \epsilon$
    \item \textbf{Stage 2:} Holding square footage and distance constant, each additional bedroom is associated with an increase of \$25,500 in house price (on average).
    \item \textbf{Stage 3:} $\widehat{\text{Price}} = 50.3 + 0.12(2000) + 25.5(3) - 8.3(5) = 50.3 + 240 + 76.5 - 41.5 = 325.3$ thousand dollars (\$325,300)
\end{itemize}

\subsection*{Problem 5: Categorical Variables, Dummy Coding \& Interactions}
\begin{itemize}
    \item \textbf{Stage 1:} In-office employees (Remote = 0) are the baseline/reference group.
    \item \textbf{Stage 2A:} $\widehat{\text{Salary}} = 45000 + 2500(5) + 7000(0) = 57,500$
    \item \textbf{Stage 2B:} $\widehat{\text{Salary}} = 45000 + 2500(5) + 7000(1) = 64,500$
    \item \textbf{Stage 3:} Remote employees earn \$7,000 more on average than in-office employees, holding years of experience constant.
    \item \textbf{Stage 4:} In Model 1, the slope is 2.5 for both groups (no interaction—parallel lines).
    \item \textbf{Stage 5A:} $\widehat{\text{Score}} = 55 + 1.8 \times \text{Hours}$ (no tutoring)
    \item \textbf{Stage 5B:} $\widehat{\text{Score}} = 55 + 1.8 \times \text{Hours} + 12 + 1.2 \times \text{Hours} = 67 + 3.0 \times \text{Hours}$ (with tutoring)
    \item \textbf{Stage 6A:} Slope = 1.8 (no tutoring), Slope = 3.0 (with tutoring). Yes, the effect differs!
    \item \textbf{Stage 6B:} $H_0: \beta_3 = 0$ (no interaction); $H_a: \beta_3 \neq 0$ (interaction exists)
\end{itemize}

\end{document}
