\documentclass[12pt]{article}
\usepackage{geometry}
\geometry{a4paper, margin=1in}
\usepackage{graphicx} % For including images
\usepackage{fancyhdr} % For headers and footers
\usepackage{hyperref}
\usepackage{caption}
\usepackage{booktabs}
\usepackage{amsmath}
\usepackage{xcolor}
\usepackage{pgfplots}
\usepackage{graphicx}
\usepackage{subcaption}
\usepackage{pdflscape}
\usepackage{float}
\usepackage{placeins}
\pgfplotsset{compat=1.17}
\usepackage{tikz}

\begin{document}

% Header block
\noindent
\begin{minipage}[t]{0.075\textwidth}
    \vspace{0pt}
    \includegraphics[width=1.6cm]{figures/Math_Mule.png} % Replace with your logo file name
\end{minipage}
\hfill
\begin{minipage}[t]{0.9\textwidth}
    \vspace{0pt}
    \begin{tabular}{p{0.55\textwidth} p{0.4\textwidth}}
        \textbf{\Large MA206} & \hfill AY25-2 \\
        Probability \& Statistics & \hfill Lesson 20 \\
        Lesson: Multiple Linear Regression II & \hfill Cadet: \underline{\hspace{2cm}}
    \end{tabular}
\end{minipage}

\vspace{1em}
\hrule
\vspace{0.5cm}

\section*{Lesson Objectives}
\begin{enumerate}
    \item Understand the impact of interaction effects on a linear model
    \item Summarize model performance in a tabular format
\end{enumerate}

\hrule
\vspace{0.5cm}



\subsection*{Interaction Effects: Conceptual Overview}

\noindent
\textbf{Motivating Example:} A tutoring program may boost low‑performers’ test scores far more than high‑performers’. This unequal benefit is an \textit{interaction} effect.

\vspace{0.25cm}

\noindent
\textbf{What Is an Interaction?}
\begin{itemize}
  \item Occurs when one predictor’s effect on \(Y\) changes based on another predictor.
  \item Captures synergy between variables.
        \begin{itemize}
            \item Amplifying
            \item Dampening
        \end{itemize}
\end{itemize}

\vspace{0.25cm}

\noindent
\textbf{Visual Comparison:}
\begin{itemize}
  \item \textbf{Before Interaction:} Parallel slopes → same effect of \(X\) across groups.
  \item \textbf{After Interaction:} Divergent slopes → effect of \(X\) depends on group.
\end{itemize}

\begin{figure}[H]
  \centering
  \includegraphics[width=0.9\textwidth]{figures/interaction_before_after.png}
  \caption{Left: No interaction. Right: With interaction.}
\end{figure}

\vspace{0.25cm}

\noindent
\textbf{General Form:} 
\[
\widehat y = \hat{\beta}_0 + \hat{\beta}_1 x_1 + \hat{\beta}_2 x_2 + \hat{\beta}_3 (x_1 \times x_2)
\]
\begin{itemize}
  \item \(\hat{\beta}_3\) $\neq$ 0 indicates an interaction may be present.
\end{itemize}

\vspace{0.25cm}

\noindent
\textbf{Interpretation of \(\hat{\beta}_3\):}
\begin{itemize}
  \item Positive: predictors amplify each other.
  \item Negative: one predictor weakens the other.
\end{itemize}

\vspace{0.5cm}



\noindent
\textbf{Why It Matters:}
\begin{itemize}
  \item Reveals conditional relationships that additive models miss.
  \item Helps identify subgroups for whom predictors have stronger or weaker effects.
  \item Improves predictive accuracy by capturing the interplay between variables.
  \item Guides targeted interventions by highlighting when and for whom a predictor is most impactful.
\end{itemize}












\section*{Interaction: Attendance × CorpsSquad}
An \textbf{interaction} means the effect of one predictor depends on another. Here, we ask: “Does the benefit of higher attendance on TEE differ for Corps Squad cadets versus non‑Corps Squad cadets?”  

\begin{figure}[H]
  \centering
  \begin{subfigure}[b]{0.45\textwidth}
    \includegraphics[width=\textwidth]{figures/attendance_interaction_plot.png}
    \caption{Predicted TEE vs Attendance}
  \end{subfigure}
  \hfill
  \begin{subfigure}[b]{0.45\textwidth}
    \includegraphics[width=\textwidth]{figures/attendance_slope_plot.png}
    \caption{Attendance Effect (Slope) by Group}
  \end{subfigure}
  \caption{Interaction shows Corps Squad cadets gain more TEE points per \% attendance than non‑Corps cadets.}
\end{figure}
\FloatBarrier

\noindent
\textbf{Takeaway:} The red line’s steeper slope indicates Corps Squad cadets benefit more from improved attendance than non‑Corps Squad Cadets — a possible interaction effect!

\subsection*{Model Summaries}
Use the package \texttt{stargazer} in R to help you make tidy tables!

% Table created by stargazer v.5.2.3 by Marek Hlavac, Social Policy Institute. E-mail: marek.hlavac at gmail.com
% Date and time: Wed, Mar 26, 2025 - 9:57:52 AM
\begin{table}[!htbp] \centering 
  \caption{Regression Results} 
  \label{} 
\begin{tabular}{@{\extracolsep{5pt}}lcccccc} 
\\[-1.8ex]\hline 
\hline \\[-1.8ex] 
 & \multicolumn{6}{c}{Dependent variable: TEE} \\ 
\cline{2-7} 
\\[-1.8ex] & \multicolumn{6}{c}{} \\ 
 & Model 1 & Model 2 & Model 3 & Model 4 & Model 5 & Model 6 \\ 
\\[-1.8ex] & (1) & (2) & (3) & (4) & (5) & (6)\\ 
\hline \\[-1.8ex] 
 gpa & 5.802 & 2.779 & 3.207 & 2.809 & 2.983 & 2.395 \\ 
  corps &  & $-$13.798$^{***}$ & $-$1.906 & 1.718 & 1.852 & $-$58.065 \\ 
  hours &  &  & 3.323$^{**}$ &  & 1.422 & $-$0.686 \\ 
  attend &  &  &  & 0.587$^{**}$ & 0.400 & 0.490$^{*}$ \\ 
  corps:attend &  &  &  &  &  & 0.825$^{*}$ \\ 
  Constant & 65.677$^{***}$ & 81.627$^{***}$ & 53.241$^{***}$ & 27.075 & 32.348 & 43.050$^{**}$ \\ 
 \hline \\[-1.8ex] 
Observations & 15 & 15 & 15 & 15 & 15 & 15 \\ 
Adjusted R$^{2}$ & 0.048 & 0.791 & 0.896 & 0.912 & 0.913 & 0.939 \\ 
\hline 
\hline \\[-1.8ex] 
\textit{Note:}  & \multicolumn{6}{r}{$^{*}$p$<$0.05; $^{**}$p$<$0.01; $^{***}$p$<$0.001} \\ 
\end{tabular} 
\end{table} 





\end{document}
